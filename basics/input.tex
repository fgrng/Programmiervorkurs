\lesson{Input und Output}

Nachdem wir ein bisschen Vertrauen in den Umgang mit \Eclipse entwickelt haben
und zumindest bereits unser erstes Programm kompiliert und ausgeführt haben,
wollen wir nun etwas spannendere Dinge tun. Nach wie vor müsst Ihr nicht jede
Zeile eures Programms verstehen. Sollte euch bei einer bestimmten Zeile trotzdem
interessieren, was genau sie tut, versucht doch eventuell sie zu entfernen, das
Programm zu kompilieren und schaut, was sich ändert.

Wir wollen uns nun mit grundlegendem Konzepten zu \emph{input} und \emph{output}
vertraut machen, denn erst wenn euer Programm mit einer Benutzerin interagiert,
wird es wirklich nützlich. Wir haben in der ersten Lektion bereits
\texttt{System.out.print} (für den \emph{Standardoutput des Systems})
kennengelernt, um Dinge auszugeben. Nun nutzen wir die \texttt{nextLine()}
Funktion eines Inputreaders, um Eingaben des Benutzers entgegen zu nehmen. Jedes
Programm unter Linux, Mac OS oder Windows kann auf diese Weise Eingaben von der
Nutzerin über die Konsole entgegen nehmen oder Ausgaben liefern. Das ist auch
der Grund, warum der Weg über die Konsole so wichtig ist und es viele Dinge
gibt, die nur mittels einer Konsole gelöst werden können: Während es viele
Stunden dauert, ein grafisches Interface zu programmieren, über die man mit dem
Programm mit der Maus kommunizieren kann, kann praktisch jeder ein textbasiertes
Konsoleninterface schreiben.

Nun aber direkt zur Praxis:

\textbf{Praxis:}
\begin{enumerate}
\item Öffnet die Datei \texttt{HelloYou.java} in \Eclipse.
\item Führt die Datei \texttt{HelloYou.java} aus. Das Programm wird Eingaben von
  Euch in der Konsole verlangen. Ihr könnt diese in der Konsole (dort wo auch
  der Text ausgegeben wird) eingeben.
\item Versucht verschiedene Eingaben an das Programm und beobachtet, was passiert.
\end{enumerate}

\javasect{inputoutput/HelloYou.java}{7}{23}

\textbf{Spiel:}

\begin{enumerate}
\item Versucht, zu verstehen, was die einzelnen Teile des Programms tun. An
  welcher Stelle erfolgt die Eingabe? Was passiert dann damit?
\item Erweitert das Programm um eigene Fragen und Ausgaben. Vergesst nicht, dass
  ihr das Programm nach jeder Änderung neu kompilieren und testen müsst.
\end{enumerate}
