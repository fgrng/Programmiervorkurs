\lesson{Der Debugger}

\textbf{Fehlerklassen}

Es ist wichtig, früh zu verstehen, dass es verschiedene Klassen von Fehlern in
einem \Java Programm gibt, die sich alle zu unterschiedlichen Zeitpunkten
auswirken. Die Hauptsächliche Klasse von Fehlern, die wir bisher betrachtet
haben, sind \emph{Compilerfehler}. Sie treten -- wie der Name nahe legt -- zur
Compilezeit auf, also wenn Ihr euer Programm in Maschinencode übersetzen wollt.
Meistens handelt es sich hier um relativ einfach erkennbare Fehler in der Syntax
des Programms (wie zum Beispiel ein vergessenes Semikolon, oder eine vergessene
geschweifte Klammer) oder um undefinierte Variablen.

Eine andere, besonders fiese Klasse von Fehlern haben wir in der letzten Lektion
kennengelernt. Wenn wir nämlich durch eine Variable teilen, und in dieser
Variable erst beim Programmlauf (zur \emph{Laufzeit}) klar wird, dass dort eine
0 steht, so tritt eine so genannte \emph{arithmetic exception} auf. Der Compiler
hat hier keine Chance, diesen Fehler zu erkennen - er weiß ja nicht, was der
Benutzer später hier eingeben wird! Da diese Klasse von Fehlern zur Laufzeit
auftritt heißen sie \emph{Laufzeitfehler}. Und sie sind immer ein Zeichen von
fundamentalen Fehlern im Programm. Sie sind also die am schwersten
aufzutreibenden Fehler, da es keine automatischen Tools gibt, die uns bei ihrer
Suche helfen.

\textbf{Der Debugger}

Wir werden (noch) nicht lernen, wie wir den Fehler aus der letzten Lektion
behandeln können, aber wir werden ein wichtiges Tool kennen lernen, um
Laufzeitfehler aufzuspüren, damit wir wenigstens wissen, wo wir mit der Lösung
anfangen können: Den \emph{Debugger} der Programmierumgebung \Eclipse.

Der Debugger ist eine Möglichkeit, unser Programm in einer besonderen Umgebung
laufen zu lassen, die es uns erlaubt es jederzeit anzuhalten, den Inhalt von
Variablen zu untersuchen oder auch Anweisung für Anweisung unser Programm vom
Computer durchgehen zu lassen.

Damit er das tun kann, braucht er vom Compiler ein paar zusätzliche
Informationen, über den Quellcode. Der Debugger muss wissen, an welcher Stelle
unseres Quellcodes er den Programmfluss unterbrechen soll, um ab dann
Schrittweise weiterzumachen und uns jeweils den Inhalt von Variablen anzuzeigen.

\javasect{debugging/Debugger.java}{7}{31}

\textbf{Praxis:}
\begin{enumerate}
\item Zu allererst müssen wir einen so genannten \emph{Breakpoint} setzen: das
  ist ein Punkt im Programmablauf, an dem vorerst gestoppt werden soll, damit
  wir entscheiden können, was wir tun wollen. Der Beginn der \texttt{main} ist
  für die meisten unserer Programme eine sichere Wahl. Durch einen Doppelklick
  auf die Zeilennummer 8 am Beginn der \texttt{main} erzeugen wir einen
  Breakpoint. Dieser wird durch einen kleinen, blauen Punkt dargestellt. Dann
  müssen wir das Programm mit \texttt{debug} (kleines Käfersymbol) -- anstatt
  wie bisher mit \texttt{run} (grünes Pfeilsymbol) --
  starten. \inputimage{debug}
\item Der Debugger wird euch jetzt den Zustand des Programms anzeigen, an dem
  ihr den Breakpoint gesetzt habt. (Sollte sich die Oberfläche von \Eclipse
  nicht automatisch ändern, meldet Euch bei Eurer Dozentin oder Eurem Dozenten.)
  Die aktuell bekannten Variablen könnt ihr im Teilfenster oben rechts
  inspizieren. \inputimage{inspect} Mit \texttt{step} könnt ihr den jeweils
  nächsten Befehl im Programm ausführen lassen. Geht das Programm Schritt für
  Schritt durch und schaut euch die Werte von \texttt{a}, \texttt{b}
  und \texttt{c} in jedem Schritt an.
\item Wenn der Debugger mit dem Programmlauf fertig ist oder ihr ihn vorzeitig
  beendet wollt, könnt ihr ihn mit \texttt{Terminate} (rotes Stoppsymbol)
  beenden. Anschließend müsst ihr in die normale \Java Ansicht zurückkehren. Das
  erreicht ihr unter \texttt{Window, Perspective, Open Perspective,
  Java}. \inputimage{returnjava}
\end{enumerate}

\textbf{Spiel:}
\begin{enumerate}
\item Ihr habt nun schon einige Programme kennen gelernt. Kompiliert und startet
  sie mit dem Debugger neu untersucht sie genauso wie obiges Programm, solange
  ihr Lust habt.
\end{enumerate}
