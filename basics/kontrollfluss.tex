\lesson{Der Kontrollfluss}

Nachdem wir etwas über die Verwendung des Debugger verstanden haben (wir werden
ihn noch häufiger benutzen), können wir uns nun wieder unserem Problem mit der
Division durch 0 zuwenden.

\javasect{arithmetik/Arith4.java}{10}{23}

Wenn wir dieses Programm kompilieren und als zweite Zahl eine 0 eingeben, werden
wir auf der Konsole ausgegeben bekommen:
\begin{minted}{text}
Gebe eine Zahl ein: 5
Gebe noch eine Zahl ein: 0
Exception in thread "main" ...
\end{minted}

Wir können das Programm auch einmal im Debugger ausführen und werden wenig
überraschend feststellen, dass die Anweisung, an der diese arithmetic exception
auftritt die ist, in der die Division steht.

Wenn wir diesen Fehler beheben wollen, haben wir eigentlich nur zwei
Möglichkeiten: Die erste ist, ihn zu ignorieren und die Schuld auf die
Benutzerin zu schieben: Warum versucht sie auch, eine 0 einzugeben? Ich hoffe,
Ihr stimmt zu, dass das nicht sehr freundlich wäre. Stellt euch vor, jedes mal,
wenn ihr in einem Programm einen Wert eingebt, auf den das Programm nicht
vorbereitet ist, würde es direkt abstürzen. Das fändet ihr vermutlich nicht so
gut, es sollte doch zumindest mal eine Fehlermeldung ausgeben und die Nutzerin
informieren, dass sie was falsch gemacht hat. Sonst macht sie es beim nächsten
Mal wieder falsch.

Und das ist der zweite Weg, den wir jetzt einschlagen wollen. Unser Programm
sollte am Besten, nachdem es die Eingabe von der Benutzerin entgegen genommen
hat, einfach überprüfen, ob die Division erlaubt ist oder nicht. Sollte die
Nutzerin eine 0 eingegeben haben, sollte sie auf den Fehler hingewiesen werden,
ansonsten sollte das Programm den Quotienten ausgeben. Diese Abhängigkeit des
Verhaltens eines Programms von den Eingaben, bezeichnen wir als
\emph{Kontrollfluss}, man kann das mit einem Diagramm verdeutlichen:

\begin{center}
    \begin{tikzpicture}[auto, node distance=3cm,>=latex']
        \tikzstyle{block} = [draw, fill=blue!20, rectangle, minimum height=3em, minimum width=6em]

        \node [block] (start) {Input};
        \node [block, right of=start] (if) { $a=0$? };
        \node [block, right of=if, node distance=4cm] (fehler) { Gib Fehler aus };
        \node [block, below of=fehler,node distance =  2cm] (quotient) { Gib Quotient aus };
        \node [block, right of=fehler, node distance = 3.5cm] (ende) { Ende };

        \draw [->] (start) -- node {} (if);
        \draw [->] (if) -- node {\texttt{ja}} (fehler);
        \draw [->] (if.south) |- node [above, near end] {\texttt{nein}} (quotient);
        \draw [->] (quotient) -| node {} (ende);
        \draw [->] (fehler) -- node {} (ende);
    \end{tikzpicture}
\end{center}

Die einfachste Möglichkeit, den Kontrollfluss zu ändern, besteht durch so
genannten „bedingten Anweisungen“:

\javasect{control/If.java}{11}{34}

In den Zeilen 21 bis 30 sehen wir, wie eine solche Bedingte Anweisung in \Java
aussieht. Wir erkennen relativ direkt unser Diagramm hier wieder: In Zeile 21
steht der „$b=0$?“ Block, in den Zeilen 22 bis 25 steht der „Gib Fehler aus“
Block und in den Zeilen 27 bis 28 der „Gib den Quotienten aus“ Block.

Beachtet allerding die doppelten Gleichheitszeichen „\verb|==|“ in Zeile
21. \Java hat getrennte Operatoren für Vergleiche und Zuweisungen -- Doppelte
Gleichheitszeichen bedeuten Vergleich („sind diese beiden gleich?“), ein
einfaches Gleichheitszeichen bedeutet Zuweisung („mache diese beiden gleich!“).

\textbf{Praxis:}
\begin{enumerate}
\item Kompiliert \texttt{If.java} für den Debugger und lasst das Programm im
  Debugger laufen. Geht Schritt für Schritt durch das Programm, mit
  verschiedenen Eingaben (wenn ihr am Ende des Programms angekommen seid, könnt
  ihr es erneut starten).
\item Nutzt Google, um herauszufinden, welche anderen Vergleichsoperatoren
  es in \Java noch gibt. Versucht, das Programm so zu verändern, dass es
  auf Ungleichheit testet, statt auf Gleichheit (sich sonst aber genauso
  verhält).
\item Wie würdet ihr testen, ob zwei Zahlen durch einander teilbar sind (Tipp:
  Ihr kennt bereits die Division mit Rest in \Java (modulo))? Schreibt ein
  Programm, welches zwei Zahlen von der Nutzerin entgegen nimmt und ausgibt, ob
  die zweite Zahl die erste teilt.
\end{enumerate}

\textbf{Spiel:}
\begin{enumerate}
\item Testet mit verschiedenen Eingaben, was passiert, wenn ihr in
  \texttt{If.java} statt zwei Gleichheitszeichen nur eines benutzt. Benutzt den
  Debugger, um euch den Inhalt von \texttt{b} vor und nach dem Test anzuschauen.
\item Schreibt ein Programm, welches die Benutzerin fragt, wie sie heißt. Gibt
  sie Euren eigenen Namen ein, soll das Programm begeistert über die
  Namensgleichheit sein, sonst die Nutzrin wie gewohnt begrüßen.
\end{enumerate}
