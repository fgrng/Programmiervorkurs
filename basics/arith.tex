\lesson{Arithmetik}

Wir haben in der vergangenen Lektion Variablen vom Typ \texttt{String}
kennengelernt. Zeichenketten zu speichern ist schon einmal ein guter Anfang,
aber wir wollen auch rechnen können, wir brauchen also mehr Typen für Variablen.

\Java unterstützt eine Unmenge an Datentypen und hat auch die Möglichkeit,
eigene zu definieren. Wir wollen uns hier nur mit den Wichtigsten beschäftigen.

Fangen wir mit dem wohl meist genutzten Datentyp an: Einem \texttt{int}, auch
„integer“ gesprochen. Dieser Typ speichert eine Ganzzahl (mit bestimmten
Grenzen, an die wir aber erst einmal nicht stoßen werden, von daher ignorieren
wir sie erst einmal frech). Mit \texttt{int}s können wir rechnen: das
funktioniert in \Java mit ganz normalen Rechenausdrücken, wie wir sie aus der
Schule kennen. Außerdem verwenden wir die bereits angetroffenen Zuweisungen:

\javasect{arithmetik/Arith1.java}{5}{25}

Wichtig ist hier, zu beachten, dass wir dem Computer ein in Reihenfolge
abzuarbeitendes Programm geben, keine Reihe von (logischen) Aussagen. Das
bedeutet in diesem konkreten Fall, dass wir etwa nicht die Aussage treffen
„\texttt{a} ist gleich 5“, sondern dass wir sagen „lasse zuerst \texttt{a} den
Wert 5 haben. Lasse dann \texttt{b} den Wert 18 haben. Lasse dann \texttt{c} den
Wert haben, der heraus kommt, wenn man den Wert von \texttt{b} vom Wert
von \texttt{a} abzieht“. Besonders deutlich wird dieser Unterschied bei einem
Beispiel dem folgenden:

\javasect{arithmetik/Arith2.java}{5}{17}

\textbf{Praxis:}
\begin{enumerate}
\item Was gibt dieses Programm aus? Überlegt es Euch zuerst und kompiliert es
  dann, um es auszuprobieren.
\end{enumerate}

Obwohl \texttt{a = a + 19} mathematisch überhaupt keinen Sinn ergibt, ist doch
klar, was passiert, wenn man sich den Quellcode eben nicht als Reihe von
Aussagen, sondern als Folge von \emph{Anweisungen} vorstellt. Das
Gleichheitszeichen bedeutet dann nicht, dass beide Seiten gleich sein sollen,
sondern dass der Wert auf der linken Seite den Wert auf der rechten Seite
annehmen soll.

Wie wir in diesem Beispiel ausserdem sehen, können wir nicht nur Strings
ausgeben, sondern auch Zahlen. \texttt{System.out.print} gibt sie in einer Form
aus, in der wir etwas damit anfangen können. Genauso können wir auch über
\texttt{readInt()} ganze Zahlen vom Benutzer entgegen nehmen:

\javasect{arithmetik/Arith3.java}{7}{23}

Langsam aber sicher tasten wir uns an nützliche Programme heran!

\textbf{Praxis:}
\begin{enumerate}[resume]
\item Schreibt ein Programm, welches von der Nutzerin zwei ganze Zahlen entgegen
  nimmt und anschließend Summe, Differenz, Produkt und Quotient ausspuckt.
\item Was fällt auf, wenn ihr z.B. 18 und 5 eingebt?
\item Findet heraus (Google ist euer Freund), wie man in \Java Division mit Rest
  durchführt und gebt den Rest zusätzlich zu den Ergebnissen der bisherigen
  Operationen mit aus\footnote{Falls ihr nicht weiterkommt, hilft euch
    vielleicht das Stichwort „modulo“ oder „modulo-operator“ weiter.}.
\item Was passiert, wenn Ihr als zweite Zahl eine 0 eingebt?
\end{enumerate}

\textbf{Spiel:}
\begin{enumerate}
\item Findet heraus, was die größte positive (und was die kleinste negative)
  Zahl ist, die ihr in einem \texttt{int} speichern könnt. Faulpelze nutzen
  Google, Lernbegierige versuchen sie experimentell zu ermitteln. Was passiert,
  wenn Ihr eine größere Zahl eingebt?
\item Wir arbeiten bisher nur mit \texttt{int}s für ganze Zahlen. Wenn wir mit
  gebrochenen Zahlen oder Dezimalzahlen rechnen wollen brauchen wir den Datentyp
  \texttt{double}. Schreibt euer Mini Rechenprogramm so um, dass es statt
  \texttt{int}s nur noch \texttt{double} benutzt und probiert es aus. Achtet
  darauf, dass es Dezimalpunkte und Dezimalkommata gibt, wenn ihr überraschende
  Ergebnisse erhaltet.
\end{enumerate}
