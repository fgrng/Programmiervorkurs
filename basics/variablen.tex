\lesson{Variablen}

Wir wollen uns jetzt Zeilen 14 bis 18 von \texttt{HelloYou.java} genauer
anschauen:

\javasect{inputoutput/HelloYou.java}{14}{18}

Bisher haben wir uns nicht sehr darum gekümmert, wie genau dieser Teil
funktioniert. Das wollen wir nun nachholen. Beginnen wir mit Zeile 14.

Was hier passiert, ist, dass eine Variable definiert wird. Eine Variable ist im
Wesentlichen ein reservierter Bereich im \emph{Hauptspeicher}, in dem ihr Daten
ablegen könnt und der einen bestimmten Namen (in diesem Fall \texttt{eingabe})
hat. Damit der Computer weiß, wie dieser Speicherbereich zu interpretieren ist
(wir erinnern uns: Der Computer kennt keine Buchstaben oder Zahlen, nur „an“ und
„aus“), hat jede Variable einen Typ (in diesen Fall \texttt{String}, was einfach
eine Folge von Buchstaben ist).

Um einen String anzulegen gibt es noch eine andere Methode: So genannte
\emph{String-literals}. Das ist z.B. das \verb|"Hallo "| in Zeile 17.
Überall, wo ihr ein String-literal benutzen könnt, könnt ihr auch eine
String-Variable benutzen (und umgekehrt).

Wir übergeben also den String \verb|"Hallo "| an die \texttt{print} Methode
von \texttt{System.out}, um es auf der Konsole auszugeben, anschließend machen
wir das gleiche mit dem String, der in der Variable \texttt{eingabe} gespeichert
ist.

Was jetzt noch fehlt ist offensichtlich Zeile 15. Das ist, wo wir tatsächlich
eine Usereingabe holen. Der Abschnitt \texttt{reader.nextLine()} liest von der
Konsole eine von der Nutzerin eingegebene Zeile ein und gibt diese weiter. Das
Gleichheitszeichen „\verb|=|“ stellen wir uns so vor, dass die Ausgabe von
der \texttt{reader.nextLine()} Methode (also die Eingabe von der Nutzerin) in
den Speicherbereich der Variable \texttt{eingabe} geschrieben wird und zwar in
dem Sinne, dass die Ausgabe als String interpretierbar sein soll. In diesen Fall
speichern wir also einen String: es wird ein String (eine Zeichenkette) gelesen
und in die Variable \texttt{eingabe} geschoben.

Statt uns Strings aus der Konsole zu ziehen, können wir ihnen auch direkt
String-Literals \emph{zuweisen}, wie es hier passiert:

\javasect{variablen/Strings.java}{10}{14}

\textbf{Praxis:}
\begin{enumerate}
\item Was passiert, wenn Ihr nach Zeile 13 eine weitere Zuweisung an
  die Variable \texttt{hallo} macht?
\item Definiert Euch eine weitere \texttt{String} Variable und lest ein
  weiteres Wort darin ein (vielleicht ein Nachname?)
\end{enumerate}

\textbf{Spiel:}
\begin{enumerate}
\item Was passiert, wenn Ihr Euch im Namen einer Variable „vertippt“?
\item Definiert euch zwei \texttt{String} Variablen, weißt ihnen irgendwelche
  String-literals zu und versucht, die Summe von beiden Strings auszugeben.
\item Was passiert, wenn Ihr eine \texttt{String} Variable definiert, Ihr aber
  nichts zuweist und dann versucht, sie auszugeben?
\end{enumerate}
