\lesson{Hello world}

Die erste Lektion beschäftigt sich alleine mit der Frage, was eigentlich eine
Programmiersprache überhaupt ist und wie wir den Computer dazu bringen können,
daraus etwas zu machen, was er ausführen kann. In guter alter
Programmiertradition tun wir das an dem simpelsten aller Programme: Einem
Programm, was einfach nur ein „Hallo Welt!“ ausgibt.

Wie bringen wir also den Computer dazu, diese Ausgabe zu generieren? Dass er
keine natürliche Sprache versteht, sollte klar sein - intern besteht er aus
lauter Transistoren (wenn ihr nicht wisst, was das ist, denkt an winzige
Schalter), die nur die Zustände „an“ und „aus“ kennen. Wir müssen also die
Anweisung „gebe Hallo Welt aus“ in ein Format übersetzen, was nur „an“ und „aus“
benutzt.

Früher wurde genau dieses Vorgehen tatsächlich direkt benutzt: meistens über
Lochkarten, die vom Computer gelesen wurden, „ein Loch“ war dann zum Beispiel
ein „an“ und „kein Loch“ war „aus“, so wurde dann das Programm in Reihen
angeordnet und jede Reihe entsprach einem Befehl oder einem Parameter für diesen
Befehl. Dieses Format nennt sich „Maschinensprache“ und ist immer noch das, was
wir heute dem Computer übergeben, nur, dass wir keine Lochkarten mehr benutzen,
sondern Dateien, in denen lange Ströme von codierten 0en und 1en stehen.

Nun kann man sich vorstellen, dass es ganz schön anstrengend ist, ein
umfangreiches Programm in 0en und 1en zu beschreiben. Deswegen benutzt man
heutzutage so genannte Hochsprachen, um Programme zu beschreiben. Wir
beschreiben also den Programmablauf in einer von Menschen lesbaren und
verstehbaren Sprache -- wir benutzen hier \Java. Die Programmbeschreibung in
\Java legen wir dabei in einer Textdatei ab, meistens hat diese die Endung
\texttt{.java}.

Diese Beschreibung des Programms übergeben wir dann an einen \emph{Compiler},
der daraus dann Maschinencode generiert, den wir wiederum dem Computer zur
Ausführung geben können. Der Compilervorgang für \Java wird von der
Entwicklungsumgebung \Eclipse recht gut versteckt, sodass wir ihn kaum direkt
wahrnehmen. Es ist dennoch wichtig zu verstehen, dass dieser Prozess abläuft.

Dem Compiler wird von \Eclipse die zu kompilierende Textdatei übergeben. Dieser
übersetzt die Textdatei in einen ausführbaren Code, der vom Computer\footnote{Um
ehrlich zu sein, wird der Code nicht direkt vom Computer abgearbeitet.
Normalerweise wird die Textdatei in Bytecode übersetzt und von der Java Runtime
Environment (JRE) ausgeführt. Den Unterschied hier zu erklären, würde aber vom
eigentlichen Thema ablenken.} abgearbeitet werden kann.

\textbf{Praxis:}
\begin{enumerate}
  \item Ladet euch die Kursmaterialiern herunter und entpackt diese auf eurem Rechner.
	\item Startet \Eclipse und ladet das Projekt \texttt{programmiervorkurs}.
  \item Öffnet die Datei \texttt{HelloWorld.java}. 
    \inputjava{inputoutput/HelloWorld.java}
  \item Führt die Datei \texttt{HelloWolrd.java} aus. Dazu klickt ihr auf das
    grüne Startsymbol am oberen Bildschirmrand.
    \inputimage{run}
\end{enumerate}

Es sollte sich in einem unteren Teilfenster die „Konsole“ befinden, wo Ihr beim
Ausführen des Programms eine Veränderung wahrnehmen solltet. 

\inputimage{console}

Wenn das alles funktioniert hat, bedeutet das, dass Eure Programmierumgebung
richtig funktioniert und der weiteren Bearbeitung dieses Kurses keine
technischen Hürden im Weg stehen sollten.

\textbf{Spiel:}

Ihr könnt nun versuchen, den Quellcode selbst zu verändern und damit ein wenig
herumzuspielen. Denkt daran, nach jeder Änderung die Datei zu speichern und neu
auszuführen.

Dinge, die Ihr ausprobieren könntet sind zum Beispiel:
\begin{enumerate}
\item Was passiert, wenn Ihr „Hello World!“ in etwas anderes ändert?
\item Was passiert, wenn Ihr die erste oder eine beliebige Zeile löscht (der
  Originalquellcode ist in diesem pdf enthalten, ihr könnt sie also später
  wieder herstellen)?
\item Was passiert, wenn ihr das „\verb|out|“ in ein „\verb|err|“ ändert?
\item Wie könnte man mehrere Sätze ausgeben? Wie könnte man mehrere Zeilen
  ausgeben?
\end{enumerate}
