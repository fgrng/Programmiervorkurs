\lesson{Fehlermeldungen und häufige Fehler}

Wenn Ihr in den vergangen Lektionen ein bisschen herumprobiert habt, wird es
Euch sicher das ein oder andere mal passiert sein, dass Euch der Compiler statt
eines funktionierenden Programms eine Riesenmenge Fehlermeldungen ausgespuckt
hat und Ihr einen Schreck bekamt und schon dachtet, ihr hättet alles kaputt
gemacht.

\texttt{Eclipse} ist bei Fehlermeldungen immer sehr hilfsbereit und gibt
euch lieber zu viel, als wenig Informationen aus. Das kann im ersten Blick ein
bisschen überwältigend wirken, aber wenn man einmal gelernt hat, wie die
Fehlermeldungen am Besten zu lesen sind, ist das alles gar nicht mehr so
schlimm.

Wir schieben deswegen eine Lektion über häufige Fehlerquellen ein und wie man
Fehlermeldungen von \Eclipse liest, um möglichst schnell die Ursache des Fehlers
zu finden.

Nehmen wir z.B. mal folgendes Programm:

\inputjava{fehlermeldungen/Fehler1.java}

Wenn wir versuchen, dieses zu kompilieren, gibt uns \texttt{Eclipe} folgendes
aus:

{\small
\begin{textcode*}{label=Run Fehler1.Java in Eclipse}
Exception in thread "main" java.lang.Error: Unresolved compilation problem:
  The method print(string) is undefined for the type System
  at fehlermeldungen.Fehler1.main(Fehler1.java:8)
\end{textcode*}
}

Wenn wir diese Fehlermeldung verstehen wollen, fangen wir immer ganz oben an,
egal wie viel Text uns der Compiler ausspucken mag. In diesem Fall sagt uns die
erste Zeile, dass ein Fehler beim Compilieren (und nicht beim Ausführen)
aufgetreten ist (\texttt{Unresolved compilation problem}). Die nächste Zeile
verrät uns, was der Fehler war: Eine Methode ist undefiniert (\texttt{method
print(String) is undefined}). Die Zeile danach verrät uns noch genauer, wo der
Fehler passiert ist oder wo \Eclipse den Fehler vermutet (\texttt{at
fehlermeldungen.Fehler1.main(Fehler1.java:8)}. Der Fehler ergab sich im
Paket \texttt{fehlermeldungen} in der Datei \texttt{Fehler1.java} in der Zeile
8.

Das gibt Euch ganz genau die Stelle an, an der der Compiler etwas an eurem Code
zu bemängeln hat. In diesen Fall ist, was der Compiler bemängelt, dass die
Methode \texttt{print} nicht im \texttt{System} definiert ist. Das sagt uns (mit
ein bisschen Erfahrung), dass wir die Definition von \texttt{print} nicht was --
nicht weiter verwunderlich ist, denn diese beiden Dinge werden
im \texttt{out}-Teil von \texttt{System} und nicht in \texttt{System} selber
definiert. Wir haben hier vergessen, dass \texttt{out} anzugeben.

Damit wissen wir jetzt auch (endlich) was wir ändern müssen. Anstatt die
Methode \texttt{print} direkt von \texttt{System} auszurufen, müssen
wir \texttt{print} von \texttt{System.out} angeben. Offenbar ist sonst unklar,
welche Ausgabemöglichkeit des Systems verwendet werden soll. (Es gibt etwa noch
die Möglichkeit \texttt{print} von \texttt{System.err} zu verwenden, um
Warnungen oder Fehlerhinweise für unser Programm gesondert auszugeben.)

Der nächste sehr häufig vorkommende Fehler ist subtiler:

\javasect{fehlermeldungen/Fehler2.java}{3}{13}

Wenn wir versuchen, dies zu kompilieren, bekommen wir vom Compiler
entgegengespuckt:

{\small
\begin{textcode*}{label=Run Fehler1.Java in Eclipse}
Exception in thread "main" java.lang.Error: Unresolved compilation problem:
  Syntax error, insert ";" to complete BlockStatements
  at fehlermeldungen.Fehler1.main(Fehler2.java:8)
\end{textcode*}
}

Wiederum sagt uns die erste Zeile, was für ein Fehler aufgetreten ist. Die
zweite Zeile liefert uns eine genauere Fehlerbeschreibung. Die dritte Zeile sagt
uns genauer wo die Fehlerquelle erwartet wird, nämlich in Zeile 8. Die
Beschwerde des Compilers ist, dass er ein Semikolon erwartet hat, aber keins
gefunden hat. Der Grund dafür ist, dass in \Java erwartet wird, dass jede
Anweisung mit einem Semikolon abgeschlossen wird. Wenn ihr euch die bisherigen
Quellcodedateien anschaut, werdet ihr feststellen, dass hinter den allermeisten
Zeilen ein solches Semikolon steht. Hier fehlt es allerdings nach der Ausgabe in
Zeile 8. Sobald wir es hinzufügen, beschwert sich der Compiler nicht mehr.

\textbf{Praxis:}
\begin{enumerate}
\item Versucht, folgende Dateien zu kompilieren und schaut euch die
  Fehlermeldung an. In welcher Zeile, in welcher Spalte liegt der Fehler? Was
  gibt euch der Compiler als Fehlermeldung aus?
  \javasect{fehlermeldungen/Fehler3.java}{3}{13}
  \javasect{fehlermeldungen/Fehler4.java}{7}{23}
\item Versucht, die aufgetretenen Fehler zu korrigieren. Bekommt ihr es hin,
  dass der Compiler sich nicht mehr beschwert und das Programm korrekt arbeitet
  (schaut euch ggf. die bisher gezeigten Quellcodes an)?
\end{enumerate}

\textbf{Spiel:}
\begin{enumerate}
\item Das folgende Programm enthält mehrere Fehler. Bekommt ihr trotzdem raus,
  welche das sind und könnt ihr sie beheben (Tipp: „Java math“ zu
  \href{http://lmgtfy.com/?q=java+math}{googlen} kann euch hier vielleicht
  weiter bringen)?
  \inputjava{fehlermeldungen/Fehler5.java}

\item Wenn ihr in den Vergangen Lektionen ein bisschen gespielt habt und
  vereinzelnd versucht habt, Dinge zu löschen, Werden euch viele Fehlermeldungen
  begegnet sein, versucht, diese zu lesen und interpretieren, was euch der
  compiler hier sagen will.
\end{enumerate}
